% Options for packages loaded elsewhere
\PassOptionsToPackage{unicode}{hyperref}
\PassOptionsToPackage{hyphens}{url}
%
\documentclass[
]{book}
\usepackage{lmodern}
\usepackage{amssymb,amsmath}
\usepackage{ifxetex,ifluatex}
\ifnum 0\ifxetex 1\fi\ifluatex 1\fi=0 % if pdftex
  \usepackage[T1]{fontenc}
  \usepackage[utf8]{inputenc}
  \usepackage{textcomp} % provide euro and other symbols
\else % if luatex or xetex
  \usepackage{unicode-math}
  \defaultfontfeatures{Scale=MatchLowercase}
  \defaultfontfeatures[\rmfamily]{Ligatures=TeX,Scale=1}
\fi
% Use upquote if available, for straight quotes in verbatim environments
\IfFileExists{upquote.sty}{\usepackage{upquote}}{}
\IfFileExists{microtype.sty}{% use microtype if available
  \usepackage[]{microtype}
  \UseMicrotypeSet[protrusion]{basicmath} % disable protrusion for tt fonts
}{}
\makeatletter
\@ifundefined{KOMAClassName}{% if non-KOMA class
  \IfFileExists{parskip.sty}{%
    \usepackage{parskip}
  }{% else
    \setlength{\parindent}{0pt}
    \setlength{\parskip}{6pt plus 2pt minus 1pt}}
}{% if KOMA class
  \KOMAoptions{parskip=half}}
\makeatother
\usepackage{xcolor}
\IfFileExists{xurl.sty}{\usepackage{xurl}}{} % add URL line breaks if available
\IfFileExists{bookmark.sty}{\usepackage{bookmark}}{\usepackage{hyperref}}
\hypersetup{
  pdftitle={A Minimal Book Example},
  pdfauthor={Yihui Xie},
  hidelinks,
  pdfcreator={LaTeX via pandoc}}
\urlstyle{same} % disable monospaced font for URLs
\usepackage{color}
\usepackage{fancyvrb}
\newcommand{\VerbBar}{|}
\newcommand{\VERB}{\Verb[commandchars=\\\{\}]}
\DefineVerbatimEnvironment{Highlighting}{Verbatim}{commandchars=\\\{\}}
% Add ',fontsize=\small' for more characters per line
\usepackage{framed}
\definecolor{shadecolor}{RGB}{248,248,248}
\newenvironment{Shaded}{\begin{snugshade}}{\end{snugshade}}
\newcommand{\AlertTok}[1]{\textcolor[rgb]{0.94,0.16,0.16}{#1}}
\newcommand{\AnnotationTok}[1]{\textcolor[rgb]{0.56,0.35,0.01}{\textbf{\textit{#1}}}}
\newcommand{\AttributeTok}[1]{\textcolor[rgb]{0.77,0.63,0.00}{#1}}
\newcommand{\BaseNTok}[1]{\textcolor[rgb]{0.00,0.00,0.81}{#1}}
\newcommand{\BuiltInTok}[1]{#1}
\newcommand{\CharTok}[1]{\textcolor[rgb]{0.31,0.60,0.02}{#1}}
\newcommand{\CommentTok}[1]{\textcolor[rgb]{0.56,0.35,0.01}{\textit{#1}}}
\newcommand{\CommentVarTok}[1]{\textcolor[rgb]{0.56,0.35,0.01}{\textbf{\textit{#1}}}}
\newcommand{\ConstantTok}[1]{\textcolor[rgb]{0.00,0.00,0.00}{#1}}
\newcommand{\ControlFlowTok}[1]{\textcolor[rgb]{0.13,0.29,0.53}{\textbf{#1}}}
\newcommand{\DataTypeTok}[1]{\textcolor[rgb]{0.13,0.29,0.53}{#1}}
\newcommand{\DecValTok}[1]{\textcolor[rgb]{0.00,0.00,0.81}{#1}}
\newcommand{\DocumentationTok}[1]{\textcolor[rgb]{0.56,0.35,0.01}{\textbf{\textit{#1}}}}
\newcommand{\ErrorTok}[1]{\textcolor[rgb]{0.64,0.00,0.00}{\textbf{#1}}}
\newcommand{\ExtensionTok}[1]{#1}
\newcommand{\FloatTok}[1]{\textcolor[rgb]{0.00,0.00,0.81}{#1}}
\newcommand{\FunctionTok}[1]{\textcolor[rgb]{0.00,0.00,0.00}{#1}}
\newcommand{\ImportTok}[1]{#1}
\newcommand{\InformationTok}[1]{\textcolor[rgb]{0.56,0.35,0.01}{\textbf{\textit{#1}}}}
\newcommand{\KeywordTok}[1]{\textcolor[rgb]{0.13,0.29,0.53}{\textbf{#1}}}
\newcommand{\NormalTok}[1]{#1}
\newcommand{\OperatorTok}[1]{\textcolor[rgb]{0.81,0.36,0.00}{\textbf{#1}}}
\newcommand{\OtherTok}[1]{\textcolor[rgb]{0.56,0.35,0.01}{#1}}
\newcommand{\PreprocessorTok}[1]{\textcolor[rgb]{0.56,0.35,0.01}{\textit{#1}}}
\newcommand{\RegionMarkerTok}[1]{#1}
\newcommand{\SpecialCharTok}[1]{\textcolor[rgb]{0.00,0.00,0.00}{#1}}
\newcommand{\SpecialStringTok}[1]{\textcolor[rgb]{0.31,0.60,0.02}{#1}}
\newcommand{\StringTok}[1]{\textcolor[rgb]{0.31,0.60,0.02}{#1}}
\newcommand{\VariableTok}[1]{\textcolor[rgb]{0.00,0.00,0.00}{#1}}
\newcommand{\VerbatimStringTok}[1]{\textcolor[rgb]{0.31,0.60,0.02}{#1}}
\newcommand{\WarningTok}[1]{\textcolor[rgb]{0.56,0.35,0.01}{\textbf{\textit{#1}}}}
\usepackage{longtable,booktabs}
% Correct order of tables after \paragraph or \subparagraph
\usepackage{etoolbox}
\makeatletter
\patchcmd\longtable{\par}{\if@noskipsec\mbox{}\fi\par}{}{}
\makeatother
% Allow footnotes in longtable head/foot
\IfFileExists{footnotehyper.sty}{\usepackage{footnotehyper}}{\usepackage{footnote}}
\makesavenoteenv{longtable}
\usepackage{graphicx}
\makeatletter
\def\maxwidth{\ifdim\Gin@nat@width>\linewidth\linewidth\else\Gin@nat@width\fi}
\def\maxheight{\ifdim\Gin@nat@height>\textheight\textheight\else\Gin@nat@height\fi}
\makeatother
% Scale images if necessary, so that they will not overflow the page
% margins by default, and it is still possible to overwrite the defaults
% using explicit options in \includegraphics[width, height, ...]{}
\setkeys{Gin}{width=\maxwidth,height=\maxheight,keepaspectratio}
% Set default figure placement to htbp
\makeatletter
\def\fps@figure{htbp}
\makeatother
\setlength{\emergencystretch}{3em} % prevent overfull lines
\providecommand{\tightlist}{%
  \setlength{\itemsep}{0pt}\setlength{\parskip}{0pt}}
\setcounter{secnumdepth}{5}
\usepackage{booktabs}
\usepackage{amsthm}
\makeatletter
\def\thm@space@setup{%
  \thm@preskip=8pt plus 2pt minus 4pt
  \thm@postskip=\thm@preskip
}
\makeatother
\ifluatex
  \usepackage{selnolig}  % disable illegal ligatures
\fi
\usepackage[]{natbib}
\bibliographystyle{apalike}

\title{A Minimal Book Example}
\author{Yihui Xie}
\date{2020-11-12}

\begin{document}
\maketitle

{
\setcounter{tocdepth}{1}
\tableofcontents
}
\hypertarget{prerequisites}{%
\chapter{Prerequisites}\label{prerequisites}}

This is a \emph{sample} book written in \textbf{Markdown}. You can use anything that Pandoc's Markdown supports, e.g., a math equation \(a^2 + b^2 = c^2\).

The \textbf{bookdown} package can be installed from CRAN or Github:

\begin{Shaded}
\begin{Highlighting}[]
\FunctionTok{install.packages}\NormalTok{(}\StringTok{"bookdown"}\NormalTok{)}
\CommentTok{\# or the development version}
\CommentTok{\# devtools::install\_github("rstudio/bookdown")}
\end{Highlighting}
\end{Shaded}

Remember each Rmd file contains one and only one chapter, and a chapter is defined by the first-level heading \texttt{\#}.

To compile this example to PDF, you need XeLaTeX. You are recommended to install TinyTeX (which includes XeLaTeX): \url{https://yihui.name/tinytex/}.

\hypertarget{intro}{%
\chapter{Introduction}\label{intro}}

You can label chapter and section titles using \texttt{\{\#label\}} after them, e.g., we can reference Chapter \ref{intro}. If you do not manually label them, there will be automatic labels anyway, e.g., Chapter \ref{methods}.

Figures and tables with captions will be placed in \texttt{figure} and \texttt{table} environments, respectively.

\begin{Shaded}
\begin{Highlighting}[]
\FunctionTok{par}\NormalTok{(}\AttributeTok{mar =} \FunctionTok{c}\NormalTok{(}\DecValTok{4}\NormalTok{, }\DecValTok{4}\NormalTok{, .}\DecValTok{1}\NormalTok{, .}\DecValTok{1}\NormalTok{))}
\FunctionTok{plot}\NormalTok{(pressure, }\AttributeTok{type =} \StringTok{\textquotesingle{}b\textquotesingle{}}\NormalTok{, }\AttributeTok{pch =} \DecValTok{19}\NormalTok{)}
\end{Highlighting}
\end{Shaded}

\begin{figure}

{\centering \includegraphics[width=0.8\linewidth]{_main_files/figure-latex/nice-fig-1} 

}

\caption{Here is a nice figure!}\label{fig:nice-fig}
\end{figure}

Reference a figure by its code chunk label with the \texttt{fig:} prefix, e.g., see Figure \ref{fig:nice-fig}. Similarly, you can reference tables generated from \texttt{knitr::kable()}, e.g., see Table \ref{tab:nice-tab}.

\begin{Shaded}
\begin{Highlighting}[]
\NormalTok{knitr}\SpecialCharTok{::}\FunctionTok{kable}\NormalTok{(}
  \FunctionTok{head}\NormalTok{(iris, }\DecValTok{20}\NormalTok{), }\AttributeTok{caption =} \StringTok{\textquotesingle{}Here is a nice table!\textquotesingle{}}\NormalTok{,}
  \AttributeTok{booktabs =} \ConstantTok{TRUE}
\NormalTok{)}
\end{Highlighting}
\end{Shaded}

\begin{table}

\caption{\label{tab:nice-tab}Here is a nice table!}
\centering
\begin{tabular}[t]{rrrrl}
\toprule
Sepal.Length & Sepal.Width & Petal.Length & Petal.Width & Species\\
\midrule
5.1 & 3.5 & 1.4 & 0.2 & setosa\\
4.9 & 3.0 & 1.4 & 0.2 & setosa\\
4.7 & 3.2 & 1.3 & 0.2 & setosa\\
4.6 & 3.1 & 1.5 & 0.2 & setosa\\
5.0 & 3.6 & 1.4 & 0.2 & setosa\\
\addlinespace
5.4 & 3.9 & 1.7 & 0.4 & setosa\\
4.6 & 3.4 & 1.4 & 0.3 & setosa\\
5.0 & 3.4 & 1.5 & 0.2 & setosa\\
4.4 & 2.9 & 1.4 & 0.2 & setosa\\
4.9 & 3.1 & 1.5 & 0.1 & setosa\\
\addlinespace
5.4 & 3.7 & 1.5 & 0.2 & setosa\\
4.8 & 3.4 & 1.6 & 0.2 & setosa\\
4.8 & 3.0 & 1.4 & 0.1 & setosa\\
4.3 & 3.0 & 1.1 & 0.1 & setosa\\
5.8 & 4.0 & 1.2 & 0.2 & setosa\\
\addlinespace
5.7 & 4.4 & 1.5 & 0.4 & setosa\\
5.4 & 3.9 & 1.3 & 0.4 & setosa\\
5.1 & 3.5 & 1.4 & 0.3 & setosa\\
5.7 & 3.8 & 1.7 & 0.3 & setosa\\
5.1 & 3.8 & 1.5 & 0.3 & setosa\\
\bottomrule
\end{tabular}
\end{table}

You can write citations, too. For example, we are using the \textbf{bookdown} package \citep{R-bookdown} in this sample book, which was built on top of R Markdown and \textbf{knitr} \citep{xie2015}.

\hypertarget{here-adding-new-test}{%
\section{Here adding new test}\label{here-adding-new-test}}

\begin{Shaded}
\begin{Highlighting}[]
\FunctionTok{library}\NormalTok{(pins)}
\FunctionTok{board\_register}\NormalTok{(}\AttributeTok{name =} \StringTok{"pins\_board"}\NormalTok{, }\AttributeTok{url =} \StringTok{"https://raw.githubusercontent.com/predictcrypto/pins/master/"}\NormalTok{, }\AttributeTok{board =} \StringTok{"datatxt"}\NormalTok{)}
\NormalTok{cryptodata }\OtherTok{\textless{}{-}} \FunctionTok{pin\_get}\NormalTok{(}\AttributeTok{name =} \StringTok{"hitBTC\_orderbook"}\NormalTok{)}
\end{Highlighting}
\end{Shaded}

Show data

\begin{Shaded}
\begin{Highlighting}[]
\NormalTok{cryptodata}
\end{Highlighting}
\end{Shaded}

\begin{verbatim}
## # A tibble: 242,218 x 27
##    pair  symbol quote_currency ask_1_price ask_1_quantity ask_2_price ask_2_quantity ask_3_price ask_3_quantity
##    <chr> <chr>  <chr>                <dbl>          <dbl>       <dbl>          <dbl>       <dbl>          <dbl>
##  1 BTCU~ BTC    USD             15685.               2     15685.           0.000140  15685.              0.108
##  2 ETHU~ ETH    USD               463.               0.4     463.           0.549       463.              1.08 
##  3 EOSU~ EOS    USD                 2.50            60         2.50        62.8           2.50         1102.   
##  4 LTCU~ LTC    USD                59.3              3.75     59.3         60            59.3             6.5  
##  5 BSVU~ BSV    USD               159.               0.6     159.          18           159.              1.38 
##  6 ADAU~ ADA    USD                 0.106          775         0.106     2635             0.106        1481    
##  7 ZECU~ ZEC    USD                58.9             10        58.9          2.6          58.9             6.48 
##  8 TRXU~ TRX    USD                 0.0250        1050         0.0250   62755             0.0250       3630    
##  9 HTUSD HT     USD                 3.66           100.        3.67      1394.            3.67         2253.   
## 10 XMRU~ XMR    USD               113.               2.65    113.           5.54        113.              2.07 
## # ... with 242,208 more rows, and 18 more variables: ask_4_price <dbl>, ask_4_quantity <dbl>, ask_5_price <dbl>,
## #   ask_5_quantity <dbl>, bid_1_price <dbl>, bid_1_quantity <dbl>, bid_2_price <dbl>, bid_2_quantity <dbl>,
## #   bid_3_price <dbl>, bid_3_quantity <dbl>, bid_4_price <dbl>, bid_4_quantity <dbl>, bid_5_price <dbl>,
## #   bid_5_quantity <dbl>, date_time_utc <dttm>, date <date>, pkDummy <chr>, pkey <chr>
\end{verbatim}

Show nested data

\begin{Shaded}
\begin{Highlighting}[]
\FunctionTok{library}\NormalTok{(tidyverse)}
\NormalTok{cryptodata }\OtherTok{\textless{}{-}} \FunctionTok{group\_by}\NormalTok{(cryptodata, symbol)}
\FunctionTok{nest}\NormalTok{(cryptodata)}
\end{Highlighting}
\end{Shaded}

\begin{verbatim}
## # A tibble: 218 x 2
## # Groups:   symbol [218]
##    symbol data                 
##    <chr>  <list>               
##  1 BTC    <tibble [2,272 x 26]>
##  2 ETH    <tibble [1,396 x 26]>
##  3 EOS    <tibble [2,225 x 26]>
##  4 LTC    <tibble [2,272 x 26]>
##  5 BSV    <tibble [1,515 x 26]>
##  6 ADA    <tibble [1,158 x 26]>
##  7 ZEC    <tibble [1,113 x 26]>
##  8 TRX    <tibble [1,493 x 26]>
##  9 HT     <tibble [2,207 x 26]>
## 10 XMR    <tibble [2,266 x 26]>
## # ... with 208 more rows
\end{verbatim}

What does DT look like

\begin{Shaded}
\begin{Highlighting}[]
\FunctionTok{library}\NormalTok{(DT)}
\NormalTok{cryptodata}
\end{Highlighting}
\end{Shaded}

\begin{verbatim}
## # A tibble: 242,218 x 27
## # Groups:   symbol [218]
##    pair  symbol quote_currency ask_1_price ask_1_quantity ask_2_price ask_2_quantity ask_3_price ask_3_quantity
##    <chr> <chr>  <chr>                <dbl>          <dbl>       <dbl>          <dbl>       <dbl>          <dbl>
##  1 BTCU~ BTC    USD             15685.               2     15685.           0.000140  15685.              0.108
##  2 ETHU~ ETH    USD               463.               0.4     463.           0.549       463.              1.08 
##  3 EOSU~ EOS    USD                 2.50            60         2.50        62.8           2.50         1102.   
##  4 LTCU~ LTC    USD                59.3              3.75     59.3         60            59.3             6.5  
##  5 BSVU~ BSV    USD               159.               0.6     159.          18           159.              1.38 
##  6 ADAU~ ADA    USD                 0.106          775         0.106     2635             0.106        1481    
##  7 ZECU~ ZEC    USD                58.9             10        58.9          2.6          58.9             6.48 
##  8 TRXU~ TRX    USD                 0.0250        1050         0.0250   62755             0.0250       3630    
##  9 HTUSD HT     USD                 3.66           100.        3.67      1394.            3.67         2253.   
## 10 XMRU~ XMR    USD               113.               2.65    113.           5.54        113.              2.07 
## # ... with 242,208 more rows, and 18 more variables: ask_4_price <dbl>, ask_4_quantity <dbl>, ask_5_price <dbl>,
## #   ask_5_quantity <dbl>, bid_1_price <dbl>, bid_1_quantity <dbl>, bid_2_price <dbl>, bid_2_quantity <dbl>,
## #   bid_3_price <dbl>, bid_3_quantity <dbl>, bid_4_price <dbl>, bid_4_quantity <dbl>, bid_5_price <dbl>,
## #   bid_5_quantity <dbl>, date_time_utc <dttm>, date <date>, pkDummy <chr>, pkey <chr>
\end{verbatim}

And nested

\begin{Shaded}
\begin{Highlighting}[]
\FunctionTok{nest}\NormalTok{(cryptodata)}
\end{Highlighting}
\end{Shaded}

\begin{verbatim}
## # A tibble: 218 x 2
## # Groups:   symbol [218]
##    symbol data                 
##    <chr>  <list>               
##  1 BTC    <tibble [2,272 x 26]>
##  2 ETH    <tibble [1,396 x 26]>
##  3 EOS    <tibble [2,225 x 26]>
##  4 LTC    <tibble [2,272 x 26]>
##  5 BSV    <tibble [1,515 x 26]>
##  6 ADA    <tibble [1,158 x 26]>
##  7 ZEC    <tibble [1,113 x 26]>
##  8 TRX    <tibble [1,493 x 26]>
##  9 HT     <tibble [2,207 x 26]>
## 10 XMR    <tibble [2,266 x 26]>
## # ... with 208 more rows
\end{verbatim}

\hypertarget{literature}{%
\chapter{Literature}\label{literature}}

Here is a review of existing methods.

\hypertarget{methods}{%
\chapter{Methods}\label{methods}}

We describe our methods in this chapter.

\hypertarget{applications}{%
\chapter{Applications}\label{applications}}

Some \emph{significant} applications are demonstrated in this chapter.

\hypertarget{example-one}{%
\section{Example one}\label{example-one}}

\hypertarget{example-two}{%
\section{Example two}\label{example-two}}

\hypertarget{final-words}{%
\chapter{Final Words}\label{final-words}}

We have finished a nice book.

  \bibliography{book.bib,packages.bib}

\end{document}
